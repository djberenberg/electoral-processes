\label{sec:intro}
Our main interests in this research are the translation of complex voter opinions into 
the voting arena and the influence of different voting systems. 

Most opinion models, such as the classic Voter model \cite{votermodel}, 
limit agent opinion space to a single dimension (i.e, there is a single issue at hand) and often binary.
While this perspective on opinion may be useful to investigate dynamic opinion propagation, it fails to encapsulate the 
universal truth that socio-political stance is derived from many opinions.
Although models for opinion-election interaction exist \cite{hoyer1974comparing}, there is a lack of rudimentary, agent-based
techniques.
Using this perspective, we see a unique opportunity to extend agent-based voter opinion dynamics to a more complex space. \\
Our contributions are 
\begin{enumerate}
\item Complex opinion, a model for voter political stance. 
\item A comparison of three models for democratic voting that make use of complex opinion.
\item A discussion on our findings and suggestions on their implications.
\end{enumerate}
%We were interested, for the modeling of general elections of the idea that our opinions are not two-dimensional and are far from binary by having voter's opinions exist on a multidimensional vector, where each dimension symbolizes a distinct but potentially important topic that one can form an opinion on.
We seek to compare three different contemporary democratic voting systems, each utilized in political systems around the world.
These voting systems are 
\begin{description}
\item[Plurality] A familiar method analogous to traditional voting in the United States in which each citizen casts a 
single vote for their ideal candidate. 
\item[Ranked-Choice] Implemented in Ireland and Maine, USA. In its simplest form, each citizen ranks each candidate from 
    most to least ideal.
\item[Approval] Each citizen casts a vote for as many candidates as they would like. Vote percentages may exceed $100\%$.
This method is commonly utilized at academic or research institutions.
\end{description}
%The voting systems were implemented were plurality (the system in most U.S. elections), ranked-choice (used in national elections in Australia), and approval voting (utilized by many academic and research associations and institutions).
In exploring each voting systems, we examine idea of campaign strategy by way of opinion transparency
by enabling candidates to expose less than all of their political views to the population.\newline
\indent Exploring complex opinion space and comparing voting systems enables evaluation on real world political systems and
could lend insight on how to improve current political systems and shed light on strategies with which the systems
can be gamed. 
\\
